\chapter{Scope}

Identify an upcoming deal making (negotiation) opportunity at work (possibly a customer deal). \\

\noindent Apply the learning in this course to create a contract for closing the deal, stating the agreement terms (assume agreement, if needed).

\chapter{use case}

The specific use case (deal making / negotiation opportunity) chosen is the \textbf{sales pitch}
with a service contract as the wanted outcome. \\

\noindent A typical sales pitch negotiation between me as the seller (of managed IT services) and 
the opposing party as the buyer (customer of the managed service provider), will be discussed. \\
 
\chapter{preparation for the sales pitch / contract negotiation}

A couple of weeks prior to the date of the upcoming sales / contract negotiation,
one of my colleagues in the sales department of my company has fixed the date
for this upcoming sales negotiation. It will be held in person on premises of
our company in Vienna. I will talk to a representative of the opposing company,
our possible future customer, who is willing to fix the deal on a 
managed service contract in the field of IT networking. The idea is that
we, as the managed service provider, will proactively manage the on-premise
network infrastructure of the customer. From installation to configuration,
management, administration, to support, everything regarding the network
is included in the contract and will be fully managed by us. \\

\noindent We have to outline, how a contract is defined:

\begin{center}
	\enquote{a binding agreement between two or more persons or parties, especially, one legally enforceable} \autocite{merriamWebsterDefinitionContract}
\end{center}

\noindent This definition means, that both parties must come to an agreement, which is then 
legally enforceable, because it is subject to the law. In Austria, we are subject to the
type of legal system, the civil law. \\

\noindent Furthermore, we outline the key elements in creating contracts:

\begin{center}
	\begin{itemize}
		\item Is there an agreement?
		\item Is there consideration?
		\item Is the agreement legal?
		\item Must the deal be in writing?
	\end{itemize}
\end{center}

\noindent In our case of a IT managed service, this means the following. 
We must come to an agreement with our customer about the managed service contract.
Especially for our managed service contract, we want to specify all details and
determining factors in a written form in the written service contract. Which will then
be subject to Austrian civil law. Consideration in our case means, that
both parties need to give something. We will proactively provide our managed service,
while our customer will pay dividends in the form of money in return, 
on a monthly basis in the form of a flat service fee. In addition to paying money,
the customer is subject to:

\begin{center}
	\begin{itemize}
		\item the provisioning of necessary cooperation free of charge
		\item bringing about necessary votes and decisions
	\end{itemize}
\end{center}









xxxxxxxxxxxxxx \\

\noindent Within the selection of the word interview lies the first problem, which needs
to be pointed out. In fact, we should not see the
\emph{job interview} as a classic interview, for which the definition
is the following: \\

\noindent An interview is a structured conversation where one participant asks questions, and the other provides answers \autocite{merriamWebsterDefinitionInterview}. \\

\noindent Instead the job interview, as part of the job application process,
should be seen as a \emph{negotiation}. \\

\begin{center}
	\noindent Negotiation is a process whereby two or
	more parties work toward an agreement.
\end{center}

% who are the negotiating parties

\noindent The negotiating parties are

\begin{itemize}
	\item job applicant (interviewee)
	\item employer, HR staff (interviewer)
\end{itemize}

% What are they negotiating about / what are they going to negotiate about?

\noindent This negotiation is interest-based. The involved parties want different things.
While I as the job applicant want things like

\begin{itemize}
	\item appropriate salary
	\item benefits
	\item friendly colleagues
\end{itemize}

\noindent the employer wants things like certain skills of the employee

\begin{itemize}
	\item formal qualifications
	\item communication skills
	\item long term commitment
\end{itemize}

% role of applicant in the negotiation

\noindent My role in the negotiation, as the applicant in the job interview,
is primarily to negotiate my key interests with my negotiation partner
(future employer). The employer might try to bring down my hopes and wants.
But I need to use negotiation techniques, in order to not sell myself under
my value. I can definitely not fall below my BATNA. 

\begin{center}
	\noindent Best Alternative To a Negotiated Agreement
\end{center}

% BATNA of applicant

\noindent To show an example, we can set a BATNA for my salary. 
My current salary is 42,000 euros per year (3,000 euros per month).
This is my BATNA. It is the value, which I will fall back to,
when the negotiation fails. \\

% BATNA of other party

\noindent Let's make up a BATNA for my potential employer. 
There is already another applicant, who has similar qualifications as me,
and he assured his negotiation with a salary of 57,000 euros per year.
This is the BATNA of my negotiation partner, the employer. \\

\noindent The lowest salary I am willing to 
accept in the negotiation is 60,000 euros per year (4,285.71 euros per month). \\

\noindent In the negotiation of the job interview, in my case,
my opposing party is using an agent, who is my potential future 
team leader. He plays the role of the agent in the negotiation,
because he needs to be used to negotiate about technical aspects
of the job. \\


% identify style of negotiation one would want to use

\noindent From theory, we know five different negotiation styles

\begin{itemize}
	\item Competing
	\item Avoiding
	\item Compromising
	\item Collaborating
	\item Accommodating
\end{itemize}

\noindent In this kind of negotiation (job interview) from the perspective 
of the applicant (interviewee), I would aim for the \emph{Collaborating} 
negotiation style. This way, both parties win and meet their requirements and
needs as the outcome. However, this requires cooperation from both parties,
and might not always be possible for every aspect. \\

% What learning in this course so far can you apply in the preparation?

\noindent I can help myself in the negotiation process, if I ask some open
questions. People in general love to talk about themselves and
their values. Also you should always listen carefully
to what your partner is saying and pick up on the ideas.
Your conversational partner will like you subconsciously. \\






















