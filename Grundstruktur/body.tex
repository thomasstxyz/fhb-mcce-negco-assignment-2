\chapter{Scope}

Identify an upcoming deal making (negotiation) opportunity at work (possibly a customer deal). \\

\noindent Apply the learning in this course to create a contract for closing the deal, stating the agreement terms (assume agreement, if needed).

\chapter{use case}

The specific use case (deal making / negotiation opportunity) chosen is the \textbf{sales pitch}
with a service contract as the wanted outcome. \\

\noindent A typical sales pitch negotiation between me as the seller (of managed IT services) and 
the opposing party as the buyer (customer of the managed service provider), will be discussed. \\
 
\chapter{preparation for the sales pitch / contract negotiation}

A couple of weeks prior to the date of the upcoming sales / contract negotiation,
one of my colleagues in the sales department of my company has fixed the date
for this upcoming sales negotiation. It will be held in person on premises of
our company in Vienna, Austria. I will talk to a representative of the opposing company,
our possible future customer, who is willing to fix the deal on a 
managed service contract in the field of IT networking. The idea is that
we, as the managed service provider, will proactively manage the on-premise
network infrastructure of the customer. From installation to configuration,
management, administration, maintenance, to support, everything regarding the network
is included in the contract and will be fully managed by us. 
Each task or activity / inputs and outputs are specifically mentioned in the service contract.
The managed service contract has a fixed start and end date, a flat fee has to be payed by the customer
every month. \\

\noindent We have to outline, how a contract is defined:

\begin{center}
	\enquote{a binding agreement between two or more persons or parties, especially, one legally enforceable} \autocite{merriamWebsterDefinitionContract}
\end{center}

\noindent This definition means, that both parties must come to an agreement, which is then 
legally enforceable, because it is subject to the local law, where our company is located. 
In Austria, we are subject to the civil law. \\

\noindent Furthermore, we outline the key elements in creating contracts:

\begin{center}
	\begin{itemize}
		\item Is there an agreement?
		\item Is there consideration?
		\item Is the agreement legal?
		\item Must the deal be in writing?
	\end{itemize}
\end{center}

\noindent In our case of a managed IT service, this means the following. 
We must come to an agreement with our customer about the managed service contract.
Especially for our managed service contract, we want to specify all details and
determining factors in a written form in the service contract. Which will then
be subject to Austrian civil law. Consideration in our case means, that
both parties need to give something. We will proactively provide our managed service,
while our customer will pay dividends in the form of money in return, 
on a monthly basis in the form of a flat service fee. In addition to paying money,
the customer is subject to:

\begin{center}
	\begin{itemize}
		\item the provisioning of necessary cooperation free of charge
		\item bringing about necessary votes and decisions
	\end{itemize}
\end{center}

\noindent My role in the negotiation, as the seller / deal closer of the managed service contract,
is mainly to state the contract terms to my negotiation partner, the buyer
(customer of the managed service contract). I then need to make I write down
all negotiated terms in a document called the memorandum of agreement, and 
assure the possible customer accepts this memorandum of agreement.
Later I will fix a more detailed and strictly defined managed service contract,
which eventually the customer needs to sign,
and which is enforceable in court.

