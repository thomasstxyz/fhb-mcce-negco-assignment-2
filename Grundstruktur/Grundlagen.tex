\chapter{Grundlagen}

\section{Allgemeine Definitionen}
%Die in dieser Formatvorlage beispielhaft enthaltenen Überschriften sind auf die im konkreten Fall tatsächlich passenden Überschriften anzupassen.
%In diesem Teil der Arbeit werden die zum eindeutigen Verständnis unbedingt erforderlichen Grundlagen und Definitionen sowie die Erklärung wichtiger Begriffe angeführt.
%Die Gliederungspunkte müssen möglichst prägnant bezeichnet werden.

%\subsection*{Cloud Broker}
%Ein sogenannter Cloud Broker ist eine vereinheitlichte Cloud-Schnittstelle. Der
%Zweck eines Cloud Brokers ist, eine standardisierte Oberfläche für verschiedene
%APIs in der Cloud zu schaffen. Es ist eine singuläre abstrakt-programmatische
%Anlaufstelle, die den gesamten Cloud-Infrastruktur-Stack sowie aufkommende
%Cloud-zentrierte Technologien über eine einheitliche Schnittstelle darstellt
%\autocite{parameswaran2009cloud}.


\section{Stand des Wissens}
%Auch die neuesten Entwicklungen und Arbeiten auf diesem Gebiet (Stand der Wissenschaft oder auch state-of-the-art) sind darzulegen, wobei diese je nach Thema auch in der 1. Gliederungsebene behandelt werden können.

%Im folgenden Kapitel wird der Stand des Wissens anhand aktueller Literatur in
%Bezug auf die in Kapitel \ref{zielsetzungundwissenschaftlichefragestellung}
%gestellte Forschungsfrage dargelegt. \\

