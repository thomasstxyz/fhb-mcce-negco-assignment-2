\documentclass{beamer}

\title{Apache und Nginx im Vergleich in ihrer Funktion als HTTP Load Balancer}
\subtitle{Bachelorarbeit}
\author{Thomas Stadler}

\usepackage[style=apa, backend=biber, language=ngerman]{biblatex}
\addbibresource{References.bib}
\renewcommand\bibname{\section{Literaturverzeichnis}}


\usetheme{Frankfurt}
\usepackage{xcolor}
\usepackage{textpos}
%\usepackage{enumitem}

\usepackage{graphicx}
\usepackage{subcaption}
\captionsetup{compatibility=false}

\definecolor{fhgreen}{rgb}{0,0.47,0.32}
%\definecolor{fhgreen}{rgb}{0,121,82}
\definecolor{orange}{rgb}{1,0.5,0}

\setbeamercolor{frametitle}{bg=fhgreen}
\setbeamercolor{title}{bg=fhgreen}
\setbeamercolor{itemize items}{bg=fhgreen}
\setbeamercolor{enumerate items}{bg=fhgreen}

\setbeamertemplate{itemize items}{\color{fhgreen}$\blacktriangleright$}
\setbeamertemplate{enumerate items}{\color{fhgreen}$\blacktriangleright$}

%\setbeamercolor{local structure}{fg=fhgreen}

\begin{document}

\addtobeamertemplate{frametitle}{}{%
\begin{textblock*}{100mm}(.85\textwidth,6.3cm)
%\begin{textblock*}{100mm}(.85\textwidth,-1cm)
\includegraphics[scale=0.2]{images/FH-burgenland-logo.png}
%\includegraphics[height=1cm,width=2cm]{images/FH-burgenland-logo.png}
\end{textblock*}}

% title page
\maketitle

\section{Einleitung}

\begin{frame}
    \frametitle{Problemstellung}
    \begin{itemize}
        \item Load Balancer ist kritischer Knotenpunkt
        \item Ausfall führt zu Stillstand
        \item Hohe Antwortzeit = geringe Nutzerzufriedenheit
        \item Meistgenutzt: Apache u. Nginx
    \end{itemize}
\end{frame}

\begin{frame}
    \frametitle{Zielsetzung u. Fragestellung}
    \textbf{Forschungsfrage: }
    \begin{center}
        \large{Inwieweit unterscheidet sich die Performance zwischen Apache und Nginx in der Funktion als HTTP Load Balancer?}
    \end{center}
\end{frame}

\section{Grundlagen}

\begin{frame}
    \frametitle{Grundlagen}
    \begin{itemize}
        \item Allgemeine Definitionen
        \item Load Balancing
        \item Apache HTTP Server
        \item Nginx HTTP Server
    \end{itemize}
\end{frame}

\begin{frame}
    \frametitle{Stand des Wissens}
    \textbf{Literaturrecherche}
    \begin{itemize}
        \item Apache: gut bei dynamischem Inhalt, hohe Kompatibilität
        \item Nginx: effiziente Ressourcennutzung u. starke Performance
        \item wenige Untersuchungen in der Verwendung als Load Balancer
    \end{itemize}
\end{frame}

\section{Vorgangsweise u. Methoden}

\begin{frame}
    \frametitle{Methode}
    \textbf{Technisches Experiment}
    \begin{itemize}
        \item virtualisierte Laborumgebung
    \end{itemize}
\end{frame}

\begin{frame}
    \frametitle{Vorgangsweise}
    %\setlist[enumerate]{label={\arabic*.}}
    \begin{itemize}
        \item Bestimmung der Messgrößen
    \end{itemize}
    \vspace{0.2cm}

    \begin{figure}[H]
        \centering
        \begin{subfigure}{.5\linewidth}
            \centering
            \includegraphics[scale=0.3]{images/input-output-factors-in-experiment.png}
  %\caption{}
  \label{fig:sub1}
        \end{subfigure}%
        \begin{subfigure}{.5\linewidth}
            \centering
            \includegraphics[scale=0.12]{images/120s-180s.png}
  %\caption{}
  \label{fig:sub2}
        \end{subfigure}
%\caption{byrequests vs. round-robin, 100k, 2000 Anfragen}
\label{fig:byrequestsvs.round-robin100k2000Anfragen}
    \end{figure}

    \vspace{0.2cm}
        100k bei 2000 Anfragen \\
        1M bei 500 Anfragen \\
        10M bei 50 Anfragen
\end{frame}

\begin{frame}
    \frametitle{Vorgangsweise}
    \begin{itemize}
        \item Aufbau der Laborumgebung
    \end{itemize}
    \vspace{1cm}
    \includegraphics[scale=0.48]{images/drawio/Virtualisierungsebenen.png}
\end{frame}

\begin{frame}
    \frametitle{Vorgangsweise}
    \begin{itemize}
        \item Durchführung des Experiments
    \end{itemize}
    \vspace{1cm}
    \includegraphics[scale=0.48]{images/drawio/HTTP-Anfrage.png}
\end{frame}

\begin{frame}
    \frametitle{Vorgangsweise}
    \begin{itemize}
        \item Darstellung der Ergebnisse
    \end{itemize}
    \vspace{0.3cm}
    \includegraphics[scale=0.19]{images/default-10M-req.png}
\end{frame}

\begin{frame}
    \frametitle{Vorgangsweise}
    \begin{itemize}
        \item Bewertung und Schlussfolgerungen
    \end{itemize}
\end{frame}

%\section{Empirischer Teil}
%
%\begin{frame}
%    \frametitle{Planung und Vorbereitung}
%\end{frame}
%
%\begin{frame}
%    \frametitle{Durchführung der Lasttests}
%\end{frame}

        %\item \citetitle{rohadi2020journalofphysics} von \cite{rohadi2020journalofphysics}

\section{Ergebnisse u. Schlussfolgerungen}

\begin{frame}
    \frametitle{Ergebnisse}
    \includegraphics[scale=0.19]{images/default-100k-req.png}
\end{frame}

\begin{frame}
    \frametitle{Ergebnisse}
    \includegraphics[scale=0.19]{images/default-10M-req.png}
\end{frame}

\begin{frame}
    \frametitle{Ergebnisse}
    \includegraphics[scale=0.22]{images/default-100k-ram.png}
\end{frame}

\begin{frame}
    \frametitle{Ergebnisse}
    \includegraphics[scale=0.19]{images/1h-10M.png}
\end{frame}

\begin{frame}
    \frametitle{Schlussfolgerungen}
    \textbf{Nginx}
    \begin{itemize}
        \item besser bei kleinen Websites u. vielen Anfragen
        \item stabile Leistung als Load Balancer
        \item Ressourcennutzung sehr effizient
    \end{itemize}
    \textbf{Apache}
    \begin{itemize}
        \item besser bei großen Websites u. wenigen Anfragen
        \item auftretende Fehler
        \item Anpassen der Konfiguration nötig
        \item Ressourcennutzung weniger effizient
    \end{itemize}
%    \vspace{1cm}
%    - Nginx bevorzugt als Load Balancer \\
%    - Apache in best. Szenarien besser
\end{frame}

%        \item  -- blablabla --

\begin{frame}
    \frametitle{Ende der Präsentation}
    Dankeschön!
\end{frame}

%% columns in beamer
%\begin{frame}
%    \frametitle{Columns}
%    \begin{columns}
%        \column{.5\textwidth}
%        1 column
%        \column{.5\textwidth}
%        2 column
%    \end{columns}
%\end{frame}

\end{document}
